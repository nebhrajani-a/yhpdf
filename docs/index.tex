% Created 2021-07-21 Wed 00:29
% Intended LaTeX compiler: xelatex
\documentclass[9pt]{report}
\usepackage{graphicx}
\usepackage{grffile}
\usepackage{longtable}
\usepackage{wrapfig}
\usepackage{rotating}
\usepackage[normalem]{ulem}
\usepackage{amsmath}
\usepackage{textcomp}
\usepackage{amssymb}
\usepackage{capt-of}
\usepackage[hidelinks]{hyperref}
\usepackage[a4paper, total={6in,9in}]{geometry}
         \usepackage{booktabs}
\usepackage{minted}
         \setminted{breaklines=true, style=default}
\usepackage{parskip}
\usepackage{sectsty}
\sectionfont{\fontsize{12}{15}\selectfont}
\subsectionfont{\fontsize{11}{11}\selectfont}
\setlength\parindent{0pt}
\usepackage{parskip}
\usepackage{pifont}
\makeatletter
\def\@makechapterhead#1{%
{\parindent \z@ \raggedright \normalfont
\ifnum \c@secnumdepth >\m@ne
\LARGE\bfseries \thechapter:
\fi
\interlinepenalty\@M
\LARGE \bfseries #1\par\nobreak
\vskip 10\p@
}}
\def\@makeschapterhead#1{%
{\parindent \z@ \raggedright
\normalfont
\interlinepenalty\@M
\Huge \bfseries  #1\par\nobreak
\vskip 10\p@
}}
\makeatother
\author{Nebhrajani A.V.}
\date{\today}
\title{Yash Honda PDF (\texttt{yhpdf}) Manual}
\hypersetup{
 pdfauthor={Nebhrajani A.V.},
 pdftitle={Yash Honda PDF (\texttt{yhpdf}) Manual},
 pdfkeywords={},
 pdfsubject={},
 pdfcreator={Emacs 25.2.2 (Org mode 9.3.6)},
 pdflang={English}}
\begin{document}

\maketitle
\tableofcontents


\chapter{Installation}
\label{sec:org1a53b4f}

\section{Basic}
\label{sec:orgff09986}
Use MS Windows 10.

\begin{enumerate}
\item Install \textbf{Ghostscript (64-bit)}:
\begin{enumerate}
\item Go to \uline{\url{https://www.ghostscript.com/download/gsdnld.html}}.
\item Download the file \textbf{Ghostscript 9.54.0 for Windows (64 bit)} with
the AGPL license. \uline{Do not use the 32 bit
version, it is incompatible with \texttt{yhpdf}.}
\item Run the installer with the standard steps, and click \texttt{Finish}
once done. It should launch a website, telling you Ghostscript
is installed.
\end{enumerate}
\item Get \textbf{yhpdf}:
\begin{enumerate}
\item Download \texttt{yhpdf.exe} from
\uline{\url{https://nebhrajani-a.github.io/yhpdf/yhpdf.exe}}. Your
browser may call it ``Dangerous'' since it's not a
Microsoft-verified program. Click the arrow for options and
select ``Keep''.
\item Copy \texttt{yhpdf} from your \texttt{Downloads} folder and paste it in
\texttt{C:\textbackslash{}Program Files} (This PC \(\rightarrow\) Local Disk
\(\rightarrow\) Program Files).
\item Double click on \texttt{yhpdf}.
\begin{enumerate}
\item If ``Windows protected your PC'' launches, click ``More
info'', then select ``Run anyway'' on the bottom right.
Windows will remember your preference.
\item \uline{Wait for 20 seconds}. If it doesn't
launch, double click again, and wait for 10 more seconds. On
old machines, \texttt{yhpdf} takes time to start up. Be patient, it
runs much faster than it starts.
\end{enumerate}
\item Once it starts, you should see a window with four buttons and
some text. Go down to the taskbar, right click on \texttt{yhpdf}'s
feather icon, and select ``Pin to taskbar''.
\item \texttt{yhpdf} is now ready for use.
\item If \texttt{yhpdf} doesn't start and reports an error, make sure
Ghostscript is installed. If there's some other error you
can't resolve, report it to the package maintainer.
\end{enumerate}
\end{enumerate}

\section{Advanced}
\label{sec:org8c45317}
WIP.

\chapter{Usage}
\label{sec:org8269758}

Usage is intuitive:

\begin{enumerate}
\item Click ``Select a file'' to upload the PDF you want to split and
compress.
\begin{enumerate}
\item A popup will tell you which file has been selected.
Click ``OK''.
\end{enumerate}
\item Click ``Select output folder'' to select the folder into which the
new PDFs will be put.
\begin{enumerate}
\item A popup will tell you which folder has been selected. Click
``OK''.
\end{enumerate}
\item Click ``Process PDF'' to process the PDF. This can take up to 60
seconds, depending on the speed of your machine and the size of
the input PDF.
\begin{enumerate}
\item \uline{Do not close \texttt{yhpdf} even if Windows says ``Not Responding''}.
Killing \texttt{yhpdf} may damage the output. Wait for it to
complete.
\item Once the PDF process is complete, a popup will ask you if you
want to view the output files. Click ``Yes''.
\end{enumerate}
\item Process as many files as you want to by selecting them and the
output folder.
\item Click ``Exit'' when you're done.
\end{enumerate}


\chapter{Technical Description}
\label{sec:orgb544c33}
\section{What?}
\label{sec:org964f28e}
\texttt{yhpdf} is a GUI tool which:
\begin{itemize}
\item Takes an input PDF file.
\item Splits it into one PDF file per page.
\item Compresses each page PDF to less than 195KB.
\end{itemize}

Given the intended use-case, the only platform with first-class
support is MS Windows 10 (compiled binary provided). Other operating
systems may be supported on a per-case basis.

\section{How?}
\label{sec:orgce8e181}
\texttt{yhpdf} uses PyPDF2 to split the PDF page-wise, then uses
Ghostscript's compression on each of these pages. If the file size
is greater than 195KB, it first tries \texttt{-dPDFSETTINGS=/ebook}. If
the file size is still greater than 195KB, it tries
\texttt{-dPDFSETTINGS=/screen}. This gets most scanned pages below 200KB.
If this fails, it raises an error. In essence, \texttt{yhpdf} is just a
\texttt{for} loop which generates per-page PDFs then compresses them.

The GUI uses Tkinter and is cross-platform. If you're proficient in
Python, you can probably get \texttt{yhpdf.py} running in your own
environment by installing all dependencies.

\section{Known Issues}
\label{sec:org61fc838}
\texttt{yhpdf} is a quickly written program, and will \textbf{not} win any
software engineering competitions. It's meant to do one very
specific job well. Thus, it has many unhandled exceptions, an ugly
interface, and rather ugly code: but it works, and it works fairly
fast.

The biggest user-facing issue is the slow startup time. This is
unfortunately out of \texttt{yhpdf}'s control, since \texttt{pyinstaller} and all
other Python packagers need to decompress the single \texttt{.exe} file to
run, which takes time, especially on Windows. Screw you, Microsoft.
Moreover, it includes within it an entire Python interpreter
so\ldots{}you know how that goes.

If this bothers you, either keep it running all the time by
auto-starting it (it uses next to no RAM and CPU while idling), or
get Python and install dependencies (needless to say: advanced
users \emph{only}).

\chapter{\texttt{yhpdf.py} Source}
\label{sec:org81374a5}

\begin{minted}[]{python}
import tkinter as tk
from tkinter import ttk
from tkinter import filedialog as fd
from tkinter.messagebox import showinfo
from tkinter.messagebox import showerror
from tkinter.messagebox import askquestion
import sys
import os
from pathlib import Path
import ghostscript
from PyPDF2 import PdfFileWriter, PdfFileReader

root = tk.Tk()
root.title("[Yash Honda] PDF Splitter/Compressor")
root.geometry('480x480')

list_of_vals = {'filename': False, 'dir': False}

def split_and_compress():
    try:
        inputpdf = PdfFileReader(open(str(list_of_vals['filename']), "rb"))
    except FileNotFoundError:
        showerror(title='Error', message="File not found!")

    size_limit = 197500
    for i in range(inputpdf.numPages):
        output = PdfFileWriter()
        output.addPage(inputpdf.getPage(i))
        with open("%s/doc-page%s.pdf" % (list_of_vals['dir'] ,i), "wb") as outputStream:
            output.write(outputStream)
        if (size_limit - int(os.stat("%s/doc-page%s.pdf" % (list_of_vals['dir'], i)).st_size)) < 0:
            args = """-dCompatibilityLevel=1.4 -dNOPAUSE -dBATCH -dQUIET -sDEVICE=pdfwrite -dPDFSETTINGS=/ebook -sOutputFile=%s/document-page%s.pdf %s/doc-page%s.pdf""" % (list_of_vals['dir'], i, list_of_vals['dir'], i)
            args = args.split()
            ghostscript.Ghostscript(*args)
            os.replace("%s/document-page%s.pdf" % (list_of_vals['dir'], i), "%s/doc-page%s.pdf" % (list_of_vals['dir'], i))
            if (size_limit - int(os.stat("%s/doc-page%s.pdf" % (list_of_vals['dir'], i)).st_size)) < 0:
                args = """-dCompatibilityLevel=1.4 -dNOPAUSE -dBATCH -dQUIET -sDEVICE=pdfwrite -dPDFSETTINGS=/screen -sOutputFile=%s/document-page%s.pdf %s/doc-page%s.pdf""" % (list_of_vals['dir'], i, list_of_vals['dir'], i)
                args = args.split()
                ghostscript.Ghostscript(*args)
                os.replace("%s/document-page%s.pdf" % (list_of_vals['dir'], i), "%s/doc-page%s.pdf" % (list_of_vals['dir'], i))
            if (size_limit - int(os.stat("%s/doc-page%s.pdf" % (list_of_vals['dir'], i)).st_size)) < 0:
                showerror(title='Error', message="ERROR: I couldn't compress page %s to less than 195KB. Consider using an online compressor or re-scanning." % (i+1))
    op = askquestion(title='Success', message='Successfully processed %s. Open output folder?' % (list_of_vals['filename']))
    if op == 'yes':
        os.startfile(list_of_vals['dir'])

def select_file():
    filetypes = (
        ('PDF files', '*.pdf'),
        ('All files', '*.*')
    )

    list_of_vals['filename'] = (fd.askopenfilename(
        title='Select scanned PDF',
        initialdir = str(Path.home()),
        filetypes=filetypes))

    if list_of_vals['filename']:
        showinfo(
            title='Selected File',
            message="Selected file %s" % (list_of_vals['filename'])
        )

def select_directory():
    list_of_vals['dir'] = fd.askdirectory(title='Select output folder', initialdir = str(Path.home()))

    if list_of_vals['dir']:
        showinfo(
            title='Selected folder',
            message="Selected folder %s" % (list_of_vals['dir'])
        )


text = ttk.Label(root, text="This is an internal program. You may not copy, distribute, or sell it.")
text.pack(expand=True)

open_button = ttk.Button(root, text='Select a file', command=select_file)
open_button.pack(expand=True)


dir_button = ttk.Button(root, text="Select output folder", command=select_directory)
dir_button.pack(expand=True)

process_button = ttk.Button(root, text="Process PDF", command=split_and_compress)
process_button.pack(expand=True)

exit_button = ttk.Button(root, text="Exit", command=sys.exit)
exit_button.pack(expand=True)


text2 = ttk.Label(root, text="All rights reserved.\n(C) Yash Honda 2021")
text2.pack(expand=True)

root.mainloop()
\end{minted}
\end{document}